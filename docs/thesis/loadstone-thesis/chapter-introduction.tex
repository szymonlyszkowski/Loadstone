\chapter{Introduction} \label{chap.introduction} 
{Nowadays thanks to unlimited internet access we can gather enormously huge amount of data easily within rapid pace. The main problem which occurs is to simplify the access to the content which is expeced from user. People upload a lot of information not neccesserily mark with appropriate sign to find it again fast. Such problem lead to develop a system which would facilitate data organization. Especially categorization is expected in GIS\footnote{Geographic Information System} e.g. Web-based application enabling map visualization or PND\footnote{Portable Navigation Device}.

}


\section{Problem and Thesis Scope}
This thesis is describing complex problem of POI\footnote{Points of Interest} categorization and its possible solutions. Computer Information System which can facitlitate to solve given problem was implemented as a practical part. Document contains also description of methods which can be applied to categorize data (which in this case is plain text). 

\section{Thesis Objectives}
As main objective was to prepare ready to use java library which would enable user flexible API\footnote{Application Programmer Interface}. In this context word 'flexible' means that user can provide data in unstuctured form and obtain result as a category of given input. Such library can be imported as plain jar file without any additional sources as it is compiled with all dependencies. The only one requried source is input data which should be provided by user as a working material. 

\section{Research Method}
As mentioned before a categorization is complex problem which has to be customized according to given data. There are a lot of text processing methods to apply. Some of them were considered to be used e.g. preprocessing, feature extraction, train classifier, SVM\footnote{Support Vector Machine}, VSM\footnote{Vector Space Model}.

\section{Document Organization}
\todo{When done describe document ordering and organization (chapters)}