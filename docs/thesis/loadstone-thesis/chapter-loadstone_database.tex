\chapter{Loadstone SQLite Database}\label{chap.loadstone}

\section{Overview}
Problem of POI categorization comes for many data sources. One of them is database which is used in legacy Symbian applications which are used by blind people. Such applications are difficult to migrate fluently to newer platforms used in high-tech smartphones e.g. iOS or Android. Such data are organized in difficult to analyze text files. Examplary database files can be found on Loadstone project website. \cite{29} The purpose of this thesis was to prepare ready to use source data for Java-based platform Android. Such approach allows programmer to easily invoke java methods delivering source data in convenient way.

\section{SQLite Database}
SQLite is a RDBMS\footnote{Relational Data Base Management System} released on Public Domain license. SQLite avails using SQL queries and flexible APIs, inter alia Java. Is widely used in embedded systems, internet browsers. SQLite is widely used as an embedded part of system so it was the most suitable choice for purpose of this thesis.
\section{SQLite Praxis}
To deliver easy access solution source *.txt files were converted to SQLite database. SQLite databases give opportunity to compress data effectively and are widely used in smartphones where resources need to have limited size. To build SQLite database special bash shell script was developed:

\begin{lstlisting}[style=BASH]	
#!/bin/sh
sqlite3 $2.db <<EOS
.headers on
CREATE TABLE LoadstoneTotalDataObjectModel (
"name" TEXT,
"latitude" REAL,
"longitude" REAL,
"accuracy" REAL,
"satellites" INTEGER,
"priority" INTEGER,
"userid" INTEGER,
"id" INTEGER
);
.separator ,
.import $1 TotalData
CREATE TABLE DatabaseInfo ( version INT, createdDate DATETIME, accessedDate DATETIME);
INSERT INTO DatabaseInfo(version, createdDate, accessedDate) VALUES (1,datetime('now'),datetime('now'));
.tables
.schema TotalData
.schema DatabaseInfo
EOS
\end{lstlisting}

As first input parameter of the script is path to .txt file which contains data supposed to be converted to SQLite database. As second parameter takes resultant database name. Invoke example in bash shell:
\begin{lstlisting}[style=BASH]
chmod +x createDB.sh #makes script executable
createDB.sh /directory/to/txt/file/cala_polska.txt dataBaseNameWithoutExtention
\end{lstlisting}
To invoke script properly obviousy sqlite3 has to be installed on processing machine. 

\subsection{Usage in Java Project}
Resultant SQLite database obtained from bash shell script can be applied as a part of java project. It is one file which size is limited to 140 TB (according to SQLite standard). File applied in project is 263 MB. To obtain easy to apply for human communication with database inside java programming language additional java SQLite Database Model library was imported into project. \cite{31} This library presents fully ORM\footnote{Object Relational Mapping}-based approach which is very intuitve for java programmers. Such approach relieve programmers from SQL\footnote{Structured Query Language} query knowledge. It delivers nice to read and use data methods e.g.
\begin{lstlisting}[style=JAVA]
database.getObjectModelColumns()[0].getName()
// returns name of first column in SQLite database
database.getObjectModel(LoadstoneTotalDataObjectModel.class).getAll();
// returns all rows in database as a list of objects of LoadstoneTotalDataObjectModel class.
\end{lstlisting}
Programmer has to know only basic SQL syntax to formulate appopriate query in order to filter for desired data.
\subsection{Object Model}
As a point of interest categorization system is mainly focused on structuring data in intuitive way and sample data are provided in already defined standard, object model had to be developed. Database applied in java project is storing objects of LoadstoneTotalDataObjectModel class. This class is structuring data as: \textit{\textbf{lat\footnote{latitude}, lon\footnote{longitude}, name}} and anothers which are not important for purpose of this thesis. Addtionally DataModel interface was created. It was applied for purpose of future development of the system. Such design allows to append additional data sources e.g. OpenStreetMap\footnote{\url{https://www.openstreetmap.org/}}. If data source implements DataModel interface it can be used by API which is target of this thesis.      