\chapter{Loadstone SQLite Database}\label{chap.loadstone}

\section{Overview}
Problem of POI categorization comes for many data sources. One of them is database which is used in legacy Symbian applications which are used by blind people. Such applications are difficult to migrate fluently to newer platforms used in high-tech smartphones e.g iOS or Android. Such data are organized in difficult to analyze text files. Examplary database files can be found on Loadstone project website: \url{http://gps.zlotowicz.pl}. The purpose of this thesis was to prepare ready to use source data for Java-based platform Android. Such approach allows programmer to easily invoke java methods delivering source data in convenient way.

\section{SQLite Conversion}
To deliver easy access solution source *.txt files were converted to SQLite database. SQLite database gives opportunity to compress data effectively and are widely used in smartphones were resources need to have limited size. To build SQLite database special bash shell script had to be developed:

\begin{lstlisting}[style=BASH]	
#!/bin/sh
sqlite3 $2.db <<EOS
.headers on
CREATE TABLE LoadstoneTotalDataObjectModel (
"name" TEXT,
"latitude" REAL,
"longitude" REAL,
"accuracy" REAL,
"satellites" INTEGER,
"priority" INTEGER,
"userid" INTEGER,
"id" INTEGER
);
.separator ,
.import $1 TotalData
CREATE TABLE DatabaseInfo ( version INT, createdDate DATETIME, accessedDate DATETIME);
INSERT INTO DatabaseInfo(version, createdDate, accessedDate) VALUES (1,datetime('now'),datetime('now'));
.tables
.schema TotalData
.schema DatabaseInfo
EOS
\end{lstlisting}

As first input parameter of the script is path to .txt file which contains data supposed to be converted to SQLite database. As second parameter takes resultant database name. Invoke example in bash shell:
\begin{lstlisting}[style=BASH]
chmod +x createDB.sh #makes script executable
createDB.sh /directory/to/txt/file/cala_polska.txt dataBaseNameWithoutExtention
\end{lstlisting} 