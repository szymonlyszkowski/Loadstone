\chapter{Thesis results}

\section{Overview}
The main purpose of this thesis is to obtain categorization system for POIs. Such task is complex and difficult to execute for complex data which are present in modern word. It is very difficult to obtain desired results for each kind of data. Text processing is wide domain and requires a lot of computing power and time to return results which are readable for human. When some heuristics is applied processing effort can be decreased in both fileds of execution time and required computing power. As mentioned in previous chapters there are many methods which enable method categorization. What is more every method is aimed for different kind of input data. This can indicate that initial knowledge about input data is absolutely compulsory to achieve desired effect and effectiveness. In this chapter the results obtained from implemented information system will be presented for delivered source of data which is loadstone database.

\section{Preprocessing results}
Preprocessing is a task which does not involve high level of complexity in text analysis. It is very important to consider all combinations of \textit{\textbf{field separators}}(this therm denotes sign of space between analyzed words). Usually it is \textit{space} sign but not always. As already mentioned this may vary among input data and such signs could be \textit{tab key} or \textit{\_} sign. The best idea to apply text preprocessing is to collect statistics which would give information which phrases could be easily applied on data without any regression and conflicts on results. In case of loadstone database special bash script was prepared which is described in chapter section: \ref{analyze_db}. Results of this script gave heuristics which could be applied for preprocessing. Phrases which can be rejected from analyzed text without any bothering about categorization corectness are as follows:
\newline
\begin{itemize}
	\item \textbf{\textit{"ul"}}
	\item \textbf{\textit{"adres"}}
	\item \textbf{\textit{"."}}
\end{itemize}
There is a test suite in \textit{loadstone.api.classification.loadstone.LoadstonePreprocessingTest} class which presents execution of preprocessing on specially prepared DataModel's for this purpose. Results for analyzed texts with those paricular patterns to be rejected from input text are as follows:

\begin{algorithm}[h]
	\KwData{\textit{\textbf{"adres ul ul. fake name to be trimmed "}}}
	\KwResult{\textit{\textbf{"fake name to be trimmed"}}}
\caption{1st Example}\label{alg:1st}
\end{algorithm}
This example shows that phrases: \textit{"adres ul ul. " and " "} were succefully rejected and there is significantly less data to analyze in further categorization process.
\newline
Another example is taken directly from loadstone database. Preprocessed text is an loadstone database entry.
\begin{algorithm}[h]
	\KwData{\textit{\textbf{"adres ul. kramarska 15 warszawa rembertów"}}}
	\KwResult{\textit{\textbf{"kramarska 15 warszawa rembertów"}}}
\caption{2nd Example}\label{alg:2nd}
\end{algorithm}
As it is noticable also in this case preprocessing was successfull and did not rejected any data important from categorization point of view. Phrase: \textit{"adres ul. "} was trimmed correctly.
 
  
\section{Classification results}
\label{classification_results}
\subsection{Overview}
Classification is much more complicated task than preprocessing. It involves complex analysis of input data, possible misconceptions or inprecise heuristics. Moreover, categorization usually should be unambiguous but not always. Sometimes POI can be classified to more than one category as an example there can be considered a \textit{\textbf{church}} which can be categorized as \textit{"U - Activities of extraterritorial organisations and bodies "}(according to NACE standard notation) or may as well classified as \textit{"R - Arts, entertainment and recreation "}. However what is worth to mention is correlation between those two categories. POI \textit{\textbf{church}} cannot always be classified as the latter but always as the first one. Such cases can be complex and perfect solution is unlikely to find as well. The correctness of classification results may also vary among judging human beings.    
\subsection{Naive classifier categorization results}
As mentioned in chapter section: \ref{naive_classifier} naive classification bases on heuristics inherited from NACE standard. As heuristic for unambiguous results it takes information about frequency of given phrase in NACE category description. When one heuristic overcome in frequency second heuristic the latter is rejected. Two of more categories as result are returned when frequency is the same for each of given heuristic. When heuristic is not available in input data classification is unavailable. The following example can be presented for imaginary data:
\begin{algorithm}
	\KwData{\textit{\textbf{"Manufacturing Manufacturing Manufacturing Manufacturing Manufacturing Manufacturing administration administration administration"}}}
	\KwResult{\textit{\textbf{"C - Manufacturing "}}}
\caption{3rd Example}\label{alg:3rd}
\end{algorithm}
As heuristic \textbf{\textit{"Manufacturing"}} frequency was higher than \textbf{\textit{"administration"}} (six occurences versus three occurences) category C is returned as result for categorization system.
\newline
Another example demonstrate naive classification when heuristic frequency has the same value of occurences for each component:
\begin{algorithm}
	\KwData{\textit{\textbf{"Manufacturing Manufacturing Manufacturing administration administration administration"}}}
	\KwResult{\textit{\textbf{"C - Manufacturing ", "O - Public administration and defence; compulsory social security "}}}
\caption{4th Example}\label{alg:4th}
\end{algorithm}  
As heuristic \textbf{\textit{"Manufacturing"}} frequency was the same as heuristic \textbf{\textit{"administration"}} category C and O is returned as result for categorization system.

\subsubsection{Naive classifier for loadstone database}

As heuristic is suit to data which are definitely not included in loadstone database input data description (main issue comparison between english and polish language) naive classifier does not have significant utility during loadstone database analysis. Following example proves it:
\newline
\begin{algorithm}
	\KwData{\textit{\textbf{"bankomat i oddział bz bwk atm 24h bank ul. jana pawła ii 12 sieradz"}}}
	\KwResult{\textit{\textbf{"Not classified"}}}
	\caption{5th Example}\label{alg:5th}
\end{algorithm}  
As none heuristic exists in input data from NACE categories description "Not classified" result is returned.
\newline
\textit{Note: all examples are implemented as test suite of unit tests in class: loadstone.api.classification.NaiveClassifierTest}    
\subsection{Semisupervised categorization for loadstone database results}
Semisupervised categorization is very useful but requires much more complex analysis. It requires knowledge about heuristics included in input data. To perform semisupervised categorization loadstone bag of words was prepared previously (description chapter section: \ref{loadstone_bow}). Thanks to initial knowledge what kind of phrases could be used for categorization, results correctness is decent. Important factor is complete isolation from heuristics included in NACE standard category description (such apporach could possibly introduce regression) Following examples demonstrates performance of semisupervised categorization intended deliberately for input data derived from loadstone database:
\newline
\begin{algorithm}
	\KwData{\textit{\textbf{"bankomat i oddział bz bwk atm 24h bank ul. jana pawła ii 12 sieradz"}}}
	\KwResult{\textit{\textbf{"K - Financial and insurance activities "}}}
	\caption{6th Example}\label{alg:6th}
\end{algorithm}
Because of heuristics: \textbf{\textit{bankomat, atm, bank}} which are defined in loadstone BOW are all assigned to category K unamigious result is retreived. This sample cleary demonstartes difference between naive (example: \ref{alg:5th}) and semisupervised classification.
\newline
\begin{algorithm}
	\KwData{\textit{\textbf{"bankomat and pizza"}}}
	\KwResult{\textit{\textbf{"I - Accommodation and food service activities ", "K - Financial and insurance activities "}}}
	\caption{7th Example}\label{alg:7th}
\end{algorithm}
Because of heuristics: \textbf{\textit{bankomat, pizza}} which are defined in loadstone BOW but are assigned to different categories the result is ambiguous. Heuristics frequencies are equal so two categories are returned as categorization results.
\newline
There was also performed comparision of imaginary data indicated deliberately to NACE standard category description analyzed by using method of semisupervised categorization. The result is as follows:
\begin{algorithm}
	\KwData{\textit{\textbf{"Manufacturing Manufacturing Manufacturing Manufacturing Manufacturing Manufacturing administration administration administration"}}}
	\KwResult{\textit{\textbf{"Not classified"}}}
	\caption{8th Example}\label{alg:8th}
\end{algorithm}
In comparison to \ref{alg:3rd} obtained result is completely irrelevant to reality and does not give correct result.
\newline 
\textit{Note: all examples are implemented as test suite of unit tests in class: loadstone.api.classification.LoadstoneSemiSupervisedClassifierTest}

\section{Conclusion}
Topic of POI categorization is wide and complex domain. This domain mainly involves text processing and heuristics about possible hints which can be used to obtain expected results. As presented in chapter section \ref{classification_results} result may differ according to applied categorization method. A lot of factors need to be considered when categorization task is applied like: language of input data, model of input data (it is possible to analyze whole documents or large fragment of texts or ready to use databases), size of input data (performance in complex infromation systems is one of significant factor). It is really difficult challange to estimate what kind heuristics data can be obtained and work out system which would be versatile for every data source. Another important factor is choice of technology for such information system. Java possess high native library support and community as well \cite{24} \cite{25}. Unfortunately this involves a lot of reasearch in internet to find suitable collection which can be used in implementation. What is more it is often practice that code construction is getting more complex and difficult to maintain in future. Example of such practice is returning classification result as \textit{list of categories} instead of using construction called \textit{tuple}. Such constructions are available in programming languages like: \textbf{Scala} \cite{26} or \textbf{Python} \cite{27}. What is more syntax of those programming languages is easier to read (it is possible to ommit longline java syntax - not so huge effort is pushed on object encapsulation) and flexibility. Another important issue are external tools which could be used for easier maintenance of infromation system. Such tool could be \textit{Apache Spark} \cite{28}. It increases the perfromance of execution and is great manager for feature extraction, transformations or SVM. On the other hand this technology is quite new and requires significant effort at the beginning to use it.     